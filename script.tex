% Options for packages loaded elsewhere
\PassOptionsToPackage{unicode}{hyperref}
\PassOptionsToPackage{hyphens}{url}
%
\documentclass[
]{article}
\usepackage{lmodern}
\usepackage{amsmath}
\usepackage{ifxetex,ifluatex}
\ifnum 0\ifxetex 1\fi\ifluatex 1\fi=0 % if pdftex
  \usepackage[T1]{fontenc}
  \usepackage[utf8]{inputenc}
  \usepackage{textcomp} % provide euro and other symbols
  \usepackage{amssymb}
\else % if luatex or xetex
  \usepackage{unicode-math}
  \defaultfontfeatures{Scale=MatchLowercase}
  \defaultfontfeatures[\rmfamily]{Ligatures=TeX,Scale=1}
\fi
% Use upquote if available, for straight quotes in verbatim environments
\IfFileExists{upquote.sty}{\usepackage{upquote}}{}
\IfFileExists{microtype.sty}{% use microtype if available
  \usepackage[]{microtype}
  \UseMicrotypeSet[protrusion]{basicmath} % disable protrusion for tt fonts
}{}
\makeatletter
\@ifundefined{KOMAClassName}{% if non-KOMA class
  \IfFileExists{parskip.sty}{%
    \usepackage{parskip}
  }{% else
    \setlength{\parindent}{0pt}
    \setlength{\parskip}{6pt plus 2pt minus 1pt}}
}{% if KOMA class
  \KOMAoptions{parskip=half}}
\makeatother
\usepackage{xcolor}
\IfFileExists{xurl.sty}{\usepackage{xurl}}{} % add URL line breaks if available
\IfFileExists{bookmark.sty}{\usepackage{bookmark}}{\usepackage{hyperref}}
\hypersetup{
  pdftitle={Proyecto (Primera Entrega)},
  hidelinks,
  pdfcreator={LaTeX via pandoc}}
\urlstyle{same} % disable monospaced font for URLs
\usepackage[margin=1in]{geometry}
\usepackage{longtable,booktabs}
\usepackage{calc} % for calculating minipage widths
% Correct order of tables after \paragraph or \subparagraph
\usepackage{etoolbox}
\makeatletter
\patchcmd\longtable{\par}{\if@noskipsec\mbox{}\fi\par}{}{}
\makeatother
% Allow footnotes in longtable head/foot
\IfFileExists{footnotehyper.sty}{\usepackage{footnotehyper}}{\usepackage{footnote}}
\makesavenoteenv{longtable}
\usepackage{graphicx}
\makeatletter
\def\maxwidth{\ifdim\Gin@nat@width>\linewidth\linewidth\else\Gin@nat@width\fi}
\def\maxheight{\ifdim\Gin@nat@height>\textheight\textheight\else\Gin@nat@height\fi}
\makeatother
% Scale images if necessary, so that they will not overflow the page
% margins by default, and it is still possible to overwrite the defaults
% using explicit options in \includegraphics[width, height, ...]{}
\setkeys{Gin}{width=\maxwidth,height=\maxheight,keepaspectratio}
% Set default figure placement to htbp
\makeatletter
\def\fps@figure{htbp}
\makeatother
\setlength{\emergencystretch}{3em} % prevent overfull lines
\providecommand{\tightlist}{%
  \setlength{\itemsep}{0pt}\setlength{\parskip}{0pt}}
\setcounter{secnumdepth}{-\maxdimen} % remove section numbering
\usepackage{graphicx} \usepackage{fancyhdr} \pagestyle{fancy} \setlength\headheight{28pt} \fancyhead[R]{\includegraphics[width=6cm]{/Users/LuisCarlosGuevara/Desktop/Programacion\ /Clase_Programacion_R/logoexternado.jpg}} \fancyfoot[LE,RO]{}
\ifluatex
  \usepackage{selnolig}  % disable illegal ligatures
\fi

\title{Proyecto (Primera Entrega)}
\author{}
\date{\vspace{-2.5em}18 de Febrero, 2021}

\begin{document}
\maketitle

\hypertarget{buen-desempeuxf1o-econuxf3mico-la-clave-del-uxe9xito-para-un-rendimiento-sobresaliente-en-los-juegos-oluxedmpicos.}{%
\section{\texorpdfstring{\textbf{Buen desempeño económico, la clave del
éxito para un rendimiento sobresaliente en los juegos
olímpicos.}}{Buen desempeño económico, la clave del éxito para un rendimiento sobresaliente en los juegos olímpicos.}}\label{buen-desempeuxf1o-econuxf3mico-la-clave-del-uxe9xito-para-un-rendimiento-sobresaliente-en-los-juegos-oluxedmpicos.}}

\hypertarget{integrantes-de-grupo.}{%
\subsubsection{\texorpdfstring{\textbf{1.} Integrantes de
grupo.}{1. Integrantes de grupo.}}\label{integrantes-de-grupo.}}

\begin{itemize}
\item
  \emph{Nicolas Gonzalez}
\item
  \emph{Joan Galeano}
\item
  \emph{Alejandro Guevara}
\end{itemize}

\hypertarget{pregunta-de-investigaciuxf3n}{%
\subsubsection{\texorpdfstring{\textbf{2.} Pregunta de
investigación:}{2. Pregunta de investigación:}}\label{pregunta-de-investigaciuxf3n}}

\begin{itemize}
\tightlist
\item
  ¿Es el buen desempeño económico un factor decisivo para un rendimiento
  sobresaliente en los Juegos Olímpicos de verano?
\end{itemize}

\hypertarget{nombre-de-la-base-de-datos-seleccionada.}{%
\subsubsection{\texorpdfstring{\textbf{3.} Nombre de la base de datos
seleccionada.}{3. Nombre de la base de datos seleccionada.}}\label{nombre-de-la-base-de-datos-seleccionada.}}

\begin{itemize}
\item
  \textbf{Olympic Medals by Country 2021.}
\item
  \textbf{Banco Mundial.}

  \begin{itemize}
  \tightlist
  \item
    PIB
  \item
    Desempleo
  \item
    Inflación
  \item
    Población
  \end{itemize}
\end{itemize}

\hypertarget{entidad-que-produjo-o-produce-los-datos.}{%
\subsubsection{\texorpdfstring{\textbf{4.} Entidad que produjo o produce
los
datos.}{4. Entidad que produjo o produce los datos.}}\label{entidad-que-produjo-o-produce-los-datos.}}

\begin{itemize}
\item
  \textbf{World Population review.}
\item
  \textbf{Banco Mundial.}
\end{itemize}

\hypertarget{nuxfamero-de-variables-y-nuxfamero-de-observaciones-que-se-encuentran-en-la-base-de-datos.}{%
\subsubsection{\texorpdfstring{\textbf{5.} Número de variables y número
de observaciones que se encuentran en la base de
datos.}{5. Número de variables y número de observaciones que se encuentran en la base de datos.}}\label{nuxfamero-de-variables-y-nuxfamero-de-observaciones-que-se-encuentran-en-la-base-de-datos.}}

\begin{itemize}
\item
  Número de variables: 9.
\item
  Número de observaciones: 134.

  El buen desempeño económico de los países está dado por una
  multiplicidad de condiciones que resulta difícil mencionar a
  cabalidad. Para este trabajo de investigación, se explicará el buen
  desempeño económico desde el análisis de los agregados macroeconómicos
  fundamentales, como lo son: la Inflación, el PIB (Producto Interno
  Bruto) y la Tasa de Desempleo. De esta manera, se podrá inferir que un
  país con un nivel de desempeño óptimo desde la óptica económica será
  aquel cuyos agregados monetarios presenten valores ajustados a sus
  rangos objetivo.

  Por otro lado, se decidió incluir el tamaño poblacional de los países
  bajo el entendido de que un mayor número de personas pueden ser más
  productivas, y mejorar el desempeño económico de los países.

  Teniendo en cuenta lo anterior, se explicará el rendimiento de
  determinados países en los juegos olímpicos de verano, desde su
  desempeño económico (Inflación, Desempleo y Producto interno Bruto) y
  su tamaño poblacional. Para medir el rendimiento de los países en los
  Juegos Olímpicos, se tendrá en cuenta el total de medallas ganadas
  durante el periodo de 1950 hasta 2021.

  \begin{longtable}[]{@{}ccccccccc@{}}
  \toprule
  \endhead
  & & & & \textbf{Variables} & & & &\tabularnewline
  País & Oro & Plata & Bronce & Total medallas & Población & PIB &
  Desempleo & Inflación\tabularnewline
  \bottomrule
  \end{longtable}
\end{itemize}

\hypertarget{si-la-base-de-datos-es-transversal-serie-de-tiempo-o-longitudinal-panel.}{%
\subsubsection{\texorpdfstring{\textbf{6.} Si la base de datos es
transversal, serie de tiempo o longitudinal
(panel).}{6. Si la base de datos es transversal, serie de tiempo o longitudinal (panel).}}\label{si-la-base-de-datos-es-transversal-serie-de-tiempo-o-longitudinal-panel.}}

\begin{itemize}
\tightlist
\item
  Tipo: Corte transversal y serie de tiempo.
\end{itemize}

\hypertarget{puxe9riodo-s-que-cubre-la-base-de-datos.}{%
\subsubsection{\texorpdfstring{\textbf{7.} Périodo (s) que cubre la base
de
datos.}{7. Périodo (s) que cubre la base de datos.}}\label{puxe9riodo-s-que-cubre-la-base-de-datos.}}

\begin{itemize}
\tightlist
\item
  Periodo de estudio: 1950-2021.
\end{itemize}

\hypertarget{si-la-base-de-datos-es-de-acceso-puxfablico-o-no.}{%
\subsubsection{\texorpdfstring{\textbf{8.} Si la base de datos es de
acceso público o
no.}{8. Si la base de datos es de acceso público o no.}}\label{si-la-base-de-datos-es-de-acceso-puxfablico-o-no.}}

\begin{itemize}
\tightlist
\item
  Acceso público: Sí.
\end{itemize}

\end{document}
