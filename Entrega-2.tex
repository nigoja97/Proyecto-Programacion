% Options for packages loaded elsewhere
\PassOptionsToPackage{unicode}{hyperref}
\PassOptionsToPackage{hyphens}{url}
%
\documentclass[
]{article}
\usepackage{lmodern}
\usepackage{amsmath}
\usepackage{ifxetex,ifluatex}
\ifnum 0\ifxetex 1\fi\ifluatex 1\fi=0 % if pdftex
  \usepackage[T1]{fontenc}
  \usepackage[utf8]{inputenc}
  \usepackage{textcomp} % provide euro and other symbols
  \usepackage{amssymb}
\else % if luatex or xetex
  \usepackage{unicode-math}
  \defaultfontfeatures{Scale=MatchLowercase}
  \defaultfontfeatures[\rmfamily]{Ligatures=TeX,Scale=1}
\fi
% Use upquote if available, for straight quotes in verbatim environments
\IfFileExists{upquote.sty}{\usepackage{upquote}}{}
\IfFileExists{microtype.sty}{% use microtype if available
  \usepackage[]{microtype}
  \UseMicrotypeSet[protrusion]{basicmath} % disable protrusion for tt fonts
}{}
\makeatletter
\@ifundefined{KOMAClassName}{% if non-KOMA class
  \IfFileExists{parskip.sty}{%
    \usepackage{parskip}
  }{% else
    \setlength{\parindent}{0pt}
    \setlength{\parskip}{6pt plus 2pt minus 1pt}}
}{% if KOMA class
  \KOMAoptions{parskip=half}}
\makeatother
\usepackage{xcolor}
\IfFileExists{xurl.sty}{\usepackage{xurl}}{} % add URL line breaks if available
\IfFileExists{bookmark.sty}{\usepackage{bookmark}}{\usepackage{hyperref}}
\hypersetup{
  pdftitle={Proyecto (Segunda Entrega)},
  hidelinks,
  pdfcreator={LaTeX via pandoc}}
\urlstyle{same} % disable monospaced font for URLs
\usepackage[margin=1in]{geometry}
\usepackage{color}
\usepackage{fancyvrb}
\newcommand{\VerbBar}{|}
\newcommand{\VERB}{\Verb[commandchars=\\\{\}]}
\DefineVerbatimEnvironment{Highlighting}{Verbatim}{commandchars=\\\{\}}
% Add ',fontsize=\small' for more characters per line
\usepackage{framed}
\definecolor{shadecolor}{RGB}{248,248,248}
\newenvironment{Shaded}{\begin{snugshade}}{\end{snugshade}}
\newcommand{\AlertTok}[1]{\textcolor[rgb]{0.94,0.16,0.16}{#1}}
\newcommand{\AnnotationTok}[1]{\textcolor[rgb]{0.56,0.35,0.01}{\textbf{\textit{#1}}}}
\newcommand{\AttributeTok}[1]{\textcolor[rgb]{0.77,0.63,0.00}{#1}}
\newcommand{\BaseNTok}[1]{\textcolor[rgb]{0.00,0.00,0.81}{#1}}
\newcommand{\BuiltInTok}[1]{#1}
\newcommand{\CharTok}[1]{\textcolor[rgb]{0.31,0.60,0.02}{#1}}
\newcommand{\CommentTok}[1]{\textcolor[rgb]{0.56,0.35,0.01}{\textit{#1}}}
\newcommand{\CommentVarTok}[1]{\textcolor[rgb]{0.56,0.35,0.01}{\textbf{\textit{#1}}}}
\newcommand{\ConstantTok}[1]{\textcolor[rgb]{0.00,0.00,0.00}{#1}}
\newcommand{\ControlFlowTok}[1]{\textcolor[rgb]{0.13,0.29,0.53}{\textbf{#1}}}
\newcommand{\DataTypeTok}[1]{\textcolor[rgb]{0.13,0.29,0.53}{#1}}
\newcommand{\DecValTok}[1]{\textcolor[rgb]{0.00,0.00,0.81}{#1}}
\newcommand{\DocumentationTok}[1]{\textcolor[rgb]{0.56,0.35,0.01}{\textbf{\textit{#1}}}}
\newcommand{\ErrorTok}[1]{\textcolor[rgb]{0.64,0.00,0.00}{\textbf{#1}}}
\newcommand{\ExtensionTok}[1]{#1}
\newcommand{\FloatTok}[1]{\textcolor[rgb]{0.00,0.00,0.81}{#1}}
\newcommand{\FunctionTok}[1]{\textcolor[rgb]{0.00,0.00,0.00}{#1}}
\newcommand{\ImportTok}[1]{#1}
\newcommand{\InformationTok}[1]{\textcolor[rgb]{0.56,0.35,0.01}{\textbf{\textit{#1}}}}
\newcommand{\KeywordTok}[1]{\textcolor[rgb]{0.13,0.29,0.53}{\textbf{#1}}}
\newcommand{\NormalTok}[1]{#1}
\newcommand{\OperatorTok}[1]{\textcolor[rgb]{0.81,0.36,0.00}{\textbf{#1}}}
\newcommand{\OtherTok}[1]{\textcolor[rgb]{0.56,0.35,0.01}{#1}}
\newcommand{\PreprocessorTok}[1]{\textcolor[rgb]{0.56,0.35,0.01}{\textit{#1}}}
\newcommand{\RegionMarkerTok}[1]{#1}
\newcommand{\SpecialCharTok}[1]{\textcolor[rgb]{0.00,0.00,0.00}{#1}}
\newcommand{\SpecialStringTok}[1]{\textcolor[rgb]{0.31,0.60,0.02}{#1}}
\newcommand{\StringTok}[1]{\textcolor[rgb]{0.31,0.60,0.02}{#1}}
\newcommand{\VariableTok}[1]{\textcolor[rgb]{0.00,0.00,0.00}{#1}}
\newcommand{\VerbatimStringTok}[1]{\textcolor[rgb]{0.31,0.60,0.02}{#1}}
\newcommand{\WarningTok}[1]{\textcolor[rgb]{0.56,0.35,0.01}{\textbf{\textit{#1}}}}
\usepackage{graphicx}
\makeatletter
\def\maxwidth{\ifdim\Gin@nat@width>\linewidth\linewidth\else\Gin@nat@width\fi}
\def\maxheight{\ifdim\Gin@nat@height>\textheight\textheight\else\Gin@nat@height\fi}
\makeatother
% Scale images if necessary, so that they will not overflow the page
% margins by default, and it is still possible to overwrite the defaults
% using explicit options in \includegraphics[width, height, ...]{}
\setkeys{Gin}{width=\maxwidth,height=\maxheight,keepaspectratio}
% Set default figure placement to htbp
\makeatletter
\def\fps@figure{htbp}
\makeatother
\setlength{\emergencystretch}{3em} % prevent overfull lines
\providecommand{\tightlist}{%
  \setlength{\itemsep}{0pt}\setlength{\parskip}{0pt}}
\setcounter{secnumdepth}{-\maxdimen} % remove section numbering
\ifluatex
  \usepackage{selnolig}  % disable illegal ligatures
\fi

\title{Proyecto (Segunda Entrega)}
\author{}
\date{\vspace{-2.5em}26/3/2021}

\begin{document}
\maketitle

\hypertarget{buen-desempeuxf1o-econuxf3mico-la-clave-del-uxe9xito-para-un-rendimiento-sobresaliente-en-los-juegos-oluxedmpicos.}{%
\section{\texorpdfstring{\textbf{Buen desempeño económico, la clave del
éxito para un rendimiento sobresaliente en los juegos
olímpicos.}}{Buen desempeño económico, la clave del éxito para un rendimiento sobresaliente en los juegos olímpicos.}}\label{buen-desempeuxf1o-econuxf3mico-la-clave-del-uxe9xito-para-un-rendimiento-sobresaliente-en-los-juegos-oluxedmpicos.}}

\hypertarget{integrantes-de-grupo.}{%
\subsubsection{\texorpdfstring{\textbf{1.} Integrantes de
grupo.}{1. Integrantes de grupo.}}\label{integrantes-de-grupo.}}

\begin{itemize}
\item
  \emph{Nicolas Gonzalez}
\item
  \emph{Joan Galeano}
\item
  \emph{Alejandro Guevara}
\end{itemize}

\hypertarget{pregunta-de-investigaciuxf3n-e-hipuxf3tesis}{%
\subsubsection{\texorpdfstring{\textbf{2.} Pregunta de investigación e
hipótesis:}{2. Pregunta de investigación e hipótesis:}}\label{pregunta-de-investigaciuxf3n-e-hipuxf3tesis}}

\begin{itemize}
\item
  ¿Es el buen desempeño económico un factor decisivo para un rendimiento
  sobresaliente en los Juegos Olímpicos de verano?
\item
  oicaoisncoasncoiascinasicnasoincasoncoasincoasihfh
  0gñosdbcsjbcñobcsoñjbcduoc weñoufgweiñucsd
\end{itemize}

\hypertarget{el-documento-usa-fuentes-variadas-de-calidad-y-pertinentes.}{%
\subsubsection{\texorpdfstring{\textbf{3.} El documento usa fuentes
variadas, de calidad y
pertinentes.}{3. El documento usa fuentes variadas, de calidad y pertinentes.}}\label{el-documento-usa-fuentes-variadas-de-calidad-y-pertinentes.}}

Paquetes

\begin{Shaded}
\begin{Highlighting}[]
\FunctionTok{require}\NormalTok{(tidyverse)}
\end{Highlighting}
\end{Shaded}

\begin{verbatim}
## Loading required package: tidyverse
\end{verbatim}

\begin{verbatim}
## -- Attaching packages --------------------------------------- tidyverse 1.3.0 --
\end{verbatim}

\begin{verbatim}
## v ggplot2 3.3.3     v purrr   0.3.4
## v tibble  3.0.5     v dplyr   1.0.4
## v tidyr   1.1.2     v stringr 1.4.0
## v readr   1.4.0     v forcats 0.5.1
\end{verbatim}

\begin{verbatim}
## -- Conflicts ------------------------------------------ tidyverse_conflicts() --
## x dplyr::filter() masks stats::filter()
## x dplyr::lag()    masks stats::lag()
\end{verbatim}

\begin{Shaded}
\begin{Highlighting}[]
\FunctionTok{require}\NormalTok{(rvest)}
\end{Highlighting}
\end{Shaded}

\begin{verbatim}
## Loading required package: rvest
\end{verbatim}

\begin{verbatim}
## Warning: package 'rvest' was built under R version 4.0.4
\end{verbatim}

\begin{verbatim}
## 
## Attaching package: 'rvest'
\end{verbatim}

\begin{verbatim}
## The following object is masked from 'package:readr':
## 
##     guess_encoding
\end{verbatim}

\begin{Shaded}
\begin{Highlighting}[]
\FunctionTok{require}\NormalTok{(haven)}
\end{Highlighting}
\end{Shaded}

\begin{verbatim}
## Loading required package: haven
\end{verbatim}

\begin{Shaded}
\begin{Highlighting}[]
\FunctionTok{require}\NormalTok{(wbstats)}
\end{Highlighting}
\end{Shaded}

\begin{verbatim}
## Loading required package: wbstats
\end{verbatim}

Creacion de la base de datos

Extraemos primero los datos sobre los paises y el total de medallas
ganadas desde 1950

\begin{Shaded}
\begin{Highlighting}[]
\NormalTok{pagina }\OtherTok{\textless{}{-}}\StringTok{"http://www.olympedia.org/statistics/medal/country"}
\NormalTok{pagina\_desc }\OtherTok{\textless{}{-}} \FunctionTok{read\_html}\NormalTok{(pagina)}

\NormalTok{paises }\OtherTok{\textless{}{-}}\NormalTok{ pagina\_desc }\SpecialCharTok{\%\textgreater{}\%} \FunctionTok{html\_nodes}\NormalTok{(}\StringTok{"td:nth{-}child(1)"}\NormalTok{) }\SpecialCharTok{\%\textgreater{}\%} \FunctionTok{html\_text}\NormalTok{() }

\NormalTok{medallas }\OtherTok{\textless{}{-}}\NormalTok{ pagina\_desc }\SpecialCharTok{\%\textgreater{}\%} \FunctionTok{html\_nodes}\NormalTok{(}\StringTok{"td:nth{-}child(6)"}\NormalTok{) }\SpecialCharTok{\%\textgreater{}\%} \FunctionTok{html\_text}\NormalTok{()}
\NormalTok{medallas }\OtherTok{\textless{}{-}} \FunctionTok{as.integer}\NormalTok{(medallas)}
\end{Highlighting}
\end{Shaded}

Ahora procedemos a extraer los datos de las variables macroeconómicas

Juntamos ambos grupos de variables en un solo tibble

\begin{Shaded}
\begin{Highlighting}[]
\NormalTok{datos }\OtherTok{\textless{}{-}} \FunctionTok{tibble}\NormalTok{(paises, medallas)}
\FunctionTok{View}\NormalTok{(datos)}
\end{Highlighting}
\end{Shaded}

\hypertarget{se-identifica-una-base-de-datos-a-la-que-podruxe1n-acceder-y-que-es-pertinente-para-responder-la-pregunta-de-investigaciuxf3n-propuesta.}{%
\subsubsection{\texorpdfstring{\textbf{4.} Se identifica una base de
datos a la que podrán acceder y que es pertinente para responder la
pregunta de investigación
propuesta.}{4. Se identifica una base de datos a la que podrán acceder y que es pertinente para responder la pregunta de investigación propuesta.}}\label{se-identifica-una-base-de-datos-a-la-que-podruxe1n-acceder-y-que-es-pertinente-para-responder-la-pregunta-de-investigaciuxf3n-propuesta.}}

\hypertarget{se-identifican-las-variables-relevantes-dentro-de-la-base-de-datos.}{%
\subsubsection{\texorpdfstring{\textbf{5.} Se identifican las variables
relevantes dentro de la base de
datos.}{5. Se identifican las variables relevantes dentro de la base de datos.}}\label{se-identifican-las-variables-relevantes-dentro-de-la-base-de-datos.}}

El buen desempeño económico de los países está dado por una
multiplicidad de condiciones que resulta difícil mencionar a cabalidad.
Para este trabajo de investigación, se explicará el buen desempeño
económico desde el análisis de los agregados macroeconómicos
fundamentales, Es por esta razón que hemos decidido identificar
variables relevantes en la base de datos, como lo es el PIB y la
Población siendo estas variables independientes fundamentales que
determinan el éxito según la literatura estudiada, Cabe aclarar que se
decidió identificar el tamaño poblacional de los países bajo el
entendido de que un mayor número de personas pueden ser más productivas,
y mejorar el desempeño económico de los países.

Por otro lado, se decidió incluir e identificar la variable inflación.
Consideramos esta de gran relevancia en la investigación para determinar
el comportamiento macroeconómico de cada país que se encuentra ligado a
su vez con el PIB. En cuanto a la variable Desempleo, determinamos su
relevancia debido a su información para analizar entorno económico de
cada uno de los países.

Teniendo en cuenta lo anterior, se explicará el rendimiento de
determinados países en los juegos olímpicos de verano, desde su
desempeño económico (Inflación, Desempleo y Producto interno Bruto) y su
tamaño poblacional. Para medir el rendimiento de los países en los
Juegos Olímpicos, se tendrá en cuenta el total de medallas ganadas
durante el periodo de 1950 hasta 2021.

\hypertarget{se-identifican-los-anuxe1lisis-estaduxedsticoseconomuxe9tricos-que-se-realizaruxe1n-y-estos-se-adecuxfaan-a-la-base-de-datos-y-a-las-variables-seleccionadas.}{%
\subsubsection{\texorpdfstring{\textbf{6.} Se identifican los análisis
estadísticos/econométricos que se realizarán, y estos se adecúan a la
base de datos y a las variables
seleccionadas.}{6. Se identifican los análisis estadísticos/econométricos que se realizarán, y estos se adecúan a la base de datos y a las variables seleccionadas.}}\label{se-identifican-los-anuxe1lisis-estaduxedsticoseconomuxe9tricos-que-se-realizaruxe1n-y-estos-se-adecuxfaan-a-la-base-de-datos-y-a-las-variables-seleccionadas.}}

\end{document}
